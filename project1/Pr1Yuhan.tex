\def\CourseName{Compiler Construction}
\def\AssignmentType{Project}
\def\Number{1}
\def\Due{Monday, October 12th}
\def\Student{Yuhan Zhao}
\def\NetID{yz2487}

\documentclass[12pt]{article}
\usepackage{amsfonts}
\usepackage{amsmath}
\usepackage{lastpage}

\setlength{\oddsidemargin}{.0in}
\setlength{\evensidemargin}{.0in}
\setlength{\textwidth}{6.3in}
\setlength{\topmargin}{-0.4in}

\newcommand{\header}[1]
{
    \begin{center}
    \framebox
    {
        \vbox
        {
            \hbox to 6in {{\bf \CourseName}  \hfill Page \thepage\ / \pageref{LastPage}}
            \vspace{4mm}
            \hbox to 6in {{\hfill \AssignmentType \Number \hfill}}
            \vspace{4mm}
            \hbox to 6in {{\Large \hfill {\bf #1} \hfill}}
            \vspace{4mm}
            \hbox to 6in {{\bf Name:} \Student \space {\bf No.} \NetID \hfill {\bf Due:} \Due}
        }
    }
    \end{center}
}

\begin{document}

\header{Documentation}
\vspace{4mm}
{\bf Expression.} When assigning operator precedence and associativity I take into
account some modern browser implementations like that of Mozilla Firefox and Chrome, so
my implementation will be slightly different (and they're all listed below):

\begin{itemize}
\item[1.] Ternary operator ``a ? b : c'' is implemented as right associative. This makes
  more sense if we take a look at expressions of the form ``a ? b : c ? d : e'' which is
  intuitively interpreted as ``If a holds then this expression evaluates to b, otherwise
  if c holds then it evaluates to d, or e if c doesn't holds.'' Thus the expression should
  be grouped as ``a ? b : (c ? d : e)'' which implies the right associativity of the
  ternary operator.
\item[2.] Member access operator ``:'' and function call operator ``()'' are assigned the
  same precedence to allow two different types of expressions to be unambiguous without
  parentheses.
  \begin{itemize}
  \item[i.] ``foo().bar'', that is, foo is of type ``function'' that returns an object that
    has a member ``bar''. This would've been invalid if we had assigned ``.'' operator higher
    precedence. Same goes to line 7 of the given sample ``Fastfib.ssc''.
  \item[ii.] ``bar.foo()'', that is, bar is an object that has a member ``foo'' which is
    callable.
  \end{itemize}
\item[3.] Three assignment operators are added: ``-='', ``*='' and ``/=''.
\item[4.] Comma operator is added to have the lowest precedence and is left associative.
  It evaluates a sequence of expressions and returns the value of the last expression.
\item[5.] Interface variable initializer production ``$E \rightarrow$ \{(Id:E)*\}'' is
  removed. To enable the same feature, ``Object Literal'' is introduced.
\item[6.] ``Object literal'' initializes an object and adds members to that object according
  to all the name-value pairs defined in the curly brackets. For example, ``\{x : 4, y : 5,\}''
  returns an object which has two members {\bf x} and {bf y} with values $4$ and $5$. Note that
  an optional trailing comma is allowed by the grammar.\\
  If we are to implement a strong typing language in the future, additional type checking can
  be performed before assigning an object literal to an interface variable. The grammar also
  allows nested object literal and passing an object literal to a function call, which saves
  the programmer a lot of pain from defining temporary variables.
\end{itemize}

\vspace{4mm}
{\bf Declaration.} Some extensions are made to create a more programmer-friendly language.

\begin{itemize}
\item[1.] Variable declaration statement is extended to allow defining multiple variables
  within one statement (definitions separated by comma).
\item[2.] Anonymous function literal is introduced yet disabled (please refer to line 44 and sort
  {\bf AnonFunc}). This introduction is just for fun since the real implementation depends on function
  closure feature which requires the compiler to embed a runtime to handle dynamic memory allocation
  and garbage collection. I'm not sure if the ARM32 machine code is sufficiant to support this feature
  (and if I have time to implement that.)
\item[3.] Originally interface declaration and function declaration were both treated as statements.
  This modification was to enable declarations within function body. If we were to implement this
  modification, we would have to define the scoping of those nested declarations and to implement
  a runtime which supports function closure. Attached file ``promise.ssc'' can be parsed by uncommenting
  all disabled sorts and productions.
\end{itemize}

\vspace{4mm}
{\bf Program} The original definition of the main sort $Program \rightarrow StatementSeq$
which is commented out, allows declarations and statements to be interweaved (It also accepts
empty input as of no declarations nor statements).\\
Current version ``Pr1Yuhan.hx'' does conform to the definition which requires at least one statement.

\end{document}

